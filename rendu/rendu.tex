\documentclass{report}

\author{Rayan MERKHI}
\title{Rendu Test Technique}
\date{17-04-2024}

\begin{document}

\chapter{Intro}

Ce rapport de rendu technique a pour but de survolé tout ce qui a été effectuer
 dans le rendu.

\chapter{React-TS + Vite}

N'ayant jamais utiliser ni React ni TypeScript ni Vite, se lancer a été une des
parties les plus difficile. En effet ne sachant pas à quoi m'attendre j'ai
commencer par faire l'installation en me disant que j'allais apprendre sur le 
tas.

C'est ce qui s'est en effet passer.

Après avoir installer les outils, j'ai commencer par faire une page qui possede
une vignette et qui n'est pas trop compliquer pour commencer le temps de me 
familiriser avec la syntaxe et les outils.

Après quelque heures de reverse design du site de google, j'ai réussi a arriver
à quelque chose de fonctionnellement proche.

Cependant j'ai décider de faire également la page de mot de passe afin d'avoir
plus de ressemblance. C'est à ce moment que je me suis rendu compte que en 
implémentant les valeurs directement je me compliquais la vie. J'ai donc
utiliser des props afins de pouvoir réutiliser des modules, ce qui est le but
de React.

\chapter{reverse-Design}

N'ayant également jamais utiliser Tailwindcss j'ai également appris surle tas.
Cela a été assez simple bien que parfois perturbant au debut. Après un rapide 
set up, le style à été beaucoup travailler pour être le plus proche du site
réel.

J'ai également pris le temps de faire des petites animations qui ont été assez
interessante et qui rendre le site plus proche de la réalité. 

Le CSS et le reverse Design sont deux choses qui ne me sont que peux familière
en tant normaux mais puisque le but étais de faire le plus de ressemblance
possible, un temps particulier à été pris pour trouver les bons margins et
padding ainsi que les couleurs les plus proches.

L'animation du texte qui s'éleve lorsque l'on écrit a été assez compliquer
également à cause du fait que c'est difficile de savoir lorsqu'il y a des 
choses écrites ou pas. Cependant cela à été réglé.

\chapter{Conclusion}

Bien que nouveau à beaucoup de ces technologies je pense avoir fait un travail
de bonne qualité, avec des animations fluides et un design au plus proche de la
vraie page.

J'ai pris environ une nuit pour faire ce travail soit environ 5/7h. Cela prend
en compte l'installation, la familiarisation et le rendu final ainsi que ce
rapport.

Merci de l'opportunité

\end{document}
